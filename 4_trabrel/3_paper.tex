\subsection{Orquestación de servicios para el desarrollo de aplicaciones para big data
}

Uno de los temas importantes cuando se habla de microservicios es la orquestación de los mismos y por ello se seleccionó el artículo "Orquestación de servicios para el desarrollo de aplicaciones para big data"\cite{orquestacion}.\par

Los autores fueron introduciendo los conceptos de microservicios aclarando que los mismos permiten replicar y distribuir logrando así altos niveles de escalabilidad, se planteó la orquestación de contenedores sobre una arquitectura distribuida virtualizada.\par

Llama la atención que al igual que otros artículos fueron mencionados los conceptos de cloud computing en Internet de las cosas aplicado a smart cities, concepto que se desarrolló y se valoró alo largo del artículo. Los problemas de smart cities que se presentaron en este artículo, plantearon desafíos como el de la heterogeneidad de los dispositivos y las tecnologías involucradas, para lo cual hay que tener en cuenta la ubicuidad y la omnipresencia de los dispositivos.
Para ello se planteó una plataforma de computación escalable, puesto que la convergencia de todos ellos requieren de que sea centralizados sus grandes  volúmenes de datos.\par

Se mencionó que las características que hacen que los sistemas de cómputo convencionales no son los adecuados para big data por los tiempos de captura procesamiento y persistencia de los datos. Se hizo referencia que no se trata de recopilar datos y hacer análisis detallado, sino de hacer interpretaciones rápidas que orienten en la toma de decisiones a partir de datos de cualquier fuente, de cara a la implementación de un sistema big data como ser el de smart cities.\par

Este artículo fue seleccionado porque trató un punto base de las arquitecturas de microservicios, que es la orquestación. Se indicó que construir aplicaciones de big data tiene como principal reto la necesidad de administrar esos recursos, conocer y planificar de qué forma van a escalar.\par

Tradicionalmente las aplicaciones escalan verticalmente, donde frente a tener que resolver algún cuello de botella se dedica una mayor cantidad de recursos.
Pero en el caso de arquitectura distribuidas basadas en big data se determinó que la escalabilidad debe ser horizontal, de modo que permita distribuir la carga de trabajo dinámicamente, tanto así como paralelizar las tareas.
Se definió como escalabilidad horizontal la implicancia del balanceo de carga, aplicación de recursos, reinstalación sesión de recursos en tiempo real y optimización de su uso de forma que el usuario no perciba degradación de performance.\par

Algo que fue aclarado y que deja de evidencia la necesidad de orquestación en este tipo arquitecturas, es enfáticamente el concepto de que los microservicios son independientes.
Cada micro servicio no conoce lo que está ejecutando el otro microservicio por lo cual al tener una serie entidades independientes y al existir transacciones que dependan de todas ellas, es necesario contar con un módulo de orquestación.
Es decir una entidad o un conjunto de entidades que pueda manejar la lógica que atraviesa todos los microservicios para identificar que se está ejecutando, en qué momento y que se debe ejecutar a continuación.
Que permita conocer el estado del sistema, poder consultar el mismo, acceder a estadísticas entre otros.\par

Para desarrollar esta arquitectura de microservicios se utilizaron contenedores, de tal manera que cada microservicio se ejecuta en un contenedor el cual contiene todas sus dependencias.
A través de estos contenedores dado que los mismos son livianos y portables, además de que son independientes al hardware y al software en donde se crearon, es que se puede armar una arquitectura distribuida con ellos.
Los contenedores generan una baja sobrecarga e introducen menos overhead que las máquinas virtuales convencionales.
Se aclaró la diferencia entre la máquina virtual y el contenedor explicitando que ambos son sistemas autocontenidos que tienen como principal diferencia que una máquina virtual necesita contener todo el sistema operativo mientras que un contenedor aprovecha el sistema operativo en el cual se ejecuta.\par

El escalado horizontal se puede realizar de diferentes maneras en el artículo indicó que una de las formas de escalar horizontalmente es mediante el uso de múltiples nodos y que para lograrlo es necesario abstraerse de la complejidad de la plataforma, y esto es posible mediante el uso de clusters de contenedores, onde cada nodo en el cluster es un servidor virtual y puede tener múltiples contenedores con una semántica común,
por lo que una aplicación puede estar formada por un grupo de estos contenedores, lo cual le permite escalar a través de múltiples nodos.
Para trabajar de esta manera se necesita administrar y coordinar esos contenedores.
Y para ello es posible realizar dos tipos administración u organización, en donde se señalaron como opciones, la orquestación y la coreografía.\par

Los autores compararon los conceptos de orquestación y coreografía de servicios. Se indicó que orquestación implica que hay una entidad central que conoce todo el funcionamiento del sistema, y conoce todas las ejecuciones que están realizando todos los servicios.
Mientras que los servicios coreografiados conocen exactamente cuando ejecutar sus operaciones y con quien debe interactuar, lo cual les quita independencia y es va en contra del paradigma de microservicios.
Solamente la entidad central es la encargada de la orquestación y conoce cuál es la meta a conseguir por lo cual la orquestación se realiza mediante definiciones explícitas de las operaciones y el orden de invocación.\par

Se mencionaron una serie de productos los cuales permiten realizar la orquestación de contenedores.
Cómo ser Kubernetes, Marathon Mesos \cite{matathonMesos}, Conductor\cite{conductor}y Docker Swarm\cite{dockerSwarm}.
Como tecnología de contenerización se decidió trabajar con Docker para implementar la orquestación se eligió  Docker Swarm. Se mencionó Docker Compose\cite{dockerCompose} como una herramienta que permite armar una aplicación con tantos componentes como sean necesarios, para posteriormente orquestar la misma en el cluster.\par

Los autores explicaron que el orquestador posee una arquitectura maestro esclavo. Cabe destacar que no es una típica arquitectura maestro esclavo. En este caso las entidades no son únicas, sino que el maestro es un nodo con un conjunto de contenedores, por lo cual frente a la caída de un contenedor se levanta otro y se sigue ejecutando obteniendo así alta disponibilidad de maestros.
En cada esclavo funciona un agente que recibe las tareas y entrega al nodo maestro los informes con el avance.
Para el caso estudio se optó como forma de acceder rápidamente a los datos, replicar los mismos.
Esto es según síndico una solución en la cual se puede lograr balanceo de carga, procesamiento concurrente y paralelización.\par

El caso de estudio, se creó mediante la configuración del cluster Swarm, una estructura formada por un maestro (manager) y cuatro esclavos (workers).
Con dicha configuración se acceden a las réplicas a través de un frontend web.\par

La conclusión fue que trabajar en el desarrollo de aplicaciones de big data permite lograr escalabilidad mediante réplicas de la aplicación, lo cual es fundamental cuando se trabaja con sistemas que demandan restricciones de tiempo.
Ese escalamiento puede aumentar o disminuir de forma elástica como consecuencia de la orquestación de sistema.\par

Como critica al artículo podemos decir que no se está de acuerdo con el concepto que plante que
el uso de contenedores permite la interoperabilidad entre todos los microservicios así como la independencia de las tecnologías internas de cada microservicio.
Es considerado que se puede tener interoperabilidad entre los microservicios sin estar dentro de contenedores. La independencia de tecnologías internas de cada microservicio la determina la arquitectura de microservicios de por sí. Y como son aplicaciones que solamente deben exponer servicios a través de HTTP, los mismos pueden ser implementados en diferentes lenguajes, sobre diferentes plataformas y sobre diferentes sistemas operativos.\par