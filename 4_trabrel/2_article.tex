
\textbf{\emph{Microservice-Oriented Platform for Internet of Big
Data Analytics: A Proof of Concept}} \cite{li_microservice-oriented_2019} 
 
El articulo parte de la base que los dispositivos de internet de las cosas (su sigla en inglés IoT) generan numerosos datos desde cualquier lugar y en cualquier momento. Se plantea demostrar que no siempre es encesario centralizar y analizar los datos acumulados como plantea las implementaciones tradicionales de analisis de grandes volumentes de datos (su sigla en inglés BDA).
Se identifica que inecesariamente son trasmitidos grandes volumenes de datos hacia el sistema centralizado, pudiendo trasmitir los datos preporcesados, reducioendo asi el costo de trasmicion de los mismos. Inspirados en la infrastrucura definida por el sofware (su sigla en inglés SDI), se plantea una plataforma orientada a microservicios para partir la concepcion monolitica de BDA y crear así componentes desacoplados y reutilizables.