
\textbf{\emph{Microservice-Oriented Platform for Internet of Big
Data Analytics: A Proof of Concept}} \cite{li_microservice-oriented_2019} 
 
El artículo se basó en que los dispositivos de internet de las cosas (su sigla en inglés IoT) generan numerosos datos desde cualquier lugar y en cualquier momento.
\\
 Se plantea demostrar que no siempre es necesario centralizar y analizar los datos acumulados como plantea las implementaciones tradicionales de análisis de grandes volúmenes de datos (su sigla en inglés BDA).
\\

Se identifica que innecesariamente son trasmitidos grandes volúmenes de datos hacia el sistema centralizado, pudiendo trasmitir los datos preprocesados, reduciendo así el costo de trasmisión de los mismos. 
Inspirados en la infraestructura definida por el software (su sigla en inglés SDI), se plantea una plataforma orientada a microservicios para partir la concepción monolítica de BDA y crear así componentes desacoplados y reutilizables.
\\

Los recursos fisicos para la infrastructura fueron definidos en la medida que han sido tratados los mecanismos de soporte intermedio como capas entre los trabajos en tiempo de ejecucion y la infraestructura de los mismos.
\\
Fue indicado que  de acuerdo a las especificaciones functionales de las plataformas BDA, generalmente se combina el procesameinto de datos en conjunto con la gestion de los recursos informaticos. Incluso el almacenamiento de los datos en ocaciones son parte de la plataforma, dando lugar a un sistema que requiere un esfuerzo significane en cuento a configuracion y manipulacion. 