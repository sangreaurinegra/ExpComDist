\subsection{An Efficient and Adaptive Method for Collision
            Probability of Ships, Icebergs Using CNN
            and DBSCAN Clustering Algorithm
}

Los autores del artículo "An Efficient and Adaptive Method for Collision
Probability of Ships, Icebergs Using CNN
and DBSCAN Clustering Algorithm"\cite[pág. 20]{somaniEmerging2019}, crearon un modelo para predecir las colisiones entre barcos e icebergs,
tomando en cuenta la ubicación y velocidad del barco, y la cantidad de icebergs en la zona.
Para calcular la probabilidad de que un barco colisione con un iceberg
se utilizó el teorema de Bayes de probabilidad condicionada, analizando imágenes con la técnica CNN (Convolution neural network).
Se entrenó una inteligencia artificial (AI) utilizando CNN, mediante varios test de imágenes la AI fue aprendiendo
y mejorado la precisión.\par

Las zonas a escanear se particionaron utilizando agrupamiento DBSCAN  y se ponderaron
las mismas mediante conteo y valores condicionales. Los atributos que se suministraron al teorema de Bayes fueron
velocidad del Barco, ubicación, y presencia de icebergs (obtenida con CNN).\par

Como conclusión se destacó que el trabajo aparentó ser eficiente en predecir
una medida precisa de  las probabilidades resultantes.
Debido a que la predicción de icebergs utilizando imágenes es ineficiente se agregó una capa de una interfaz Bayesiena
para definir la probabilidad de colisión. Después de la predicción se utilizó
la técnica de agrupamiento para crear grupos que representan zonas seguras e inseguras para los barcos.\par