\paragraph{
    \textbf{\emph{An Efficient and Adaptive Method for Collision
    Probability of Ships, Icebergs Using CNN
    and DBSCAN Clustering Algorithm}
    }
    \cite[pág. 20]{somani_emerging_2019}.
}

Los autores crearon un modelo para predecir las colisiones entre barcos e icebergs, 
tomadno en cuenta la ubicación y velocidad del barco, y la cantidad de icebergs en la zona.
PAra calcular la porbabilidad de que un barco colisione con un iceberg 
se utilizó el teorema de Bayes de probabilidad condicionada.

In this proposed work, an adaptive method is used to detect the presence of icebergs and the velocity of ships, followed by integrating the obtained data and
applying the Bayesian algorithm we have successfully computed the collision
probability. This work exhibits effective results against reduced visibility due to
fog. Besides, we have acquired all the foreground authentic data from valid
resources. So, the results will help in marking the safe and unsafe zones in the
form of clusters by using DBSCAN algorithm


n this work, a new structure of detection, prediction and cluster formation is
proposed in which CNN technique is utilized along with Bayesian algorithm. CNN is
basically used to train datasets to predict whether the targeted image contains iceberg or
not. Thereafter, using those resulting attributes as inputs for Bayesian algorithm we
computed the collision probability for ships and icebergs, and lastly depending upon
the computed results we can indicate them as a safe flag or an unsafe flag

Image Classification Using CNN
Convolution neural network model works precisely on the application where image
classification is to be used. The process is simple. By passing various test images on
which the AI can train itself to the prediction accuracy which depends exponentially on
number of iterations in training the model


Bayesian Framework Inference to Predict Probability of Collision
his work uses Bayes theorem to find the collision probability of ships in iceberg prone
areas, provided that the event has already fulfilled the condition of iceberg presence and
threshold velocity


DBSCAN Clustering
We have used clustering for pointing out the safe zones in the form of clusters. The first
stage being formation of clusters with counter value or conditional value, where the
attributes are supplied to Bayesian theorem
The names are C1, C2, C3 for respectively
speed of ship, location of ship and accuracy of presence of iceberg supplied from CNN
image classifier



Step1. Input the Satellite image of glacial area [11].
Step2. Using Trained CNN Model, it will predict the probability P1 of the presence
of icebergs in specified image.
Step3. Extract the data of ships like coordinates and velocity [5].
Step4. Set Threshold value of ships [10] and icebergs for P1 probability, coordinates and velocity.
Step5. Using Bayesian algorithm, probability of collision P2 is predicted.
Step6. Applying DBSCAN algorithm on P2 to form clusters of different
probabilities.
Step7. Marking the clusters in two categories i.e., Safe and Unsafe

