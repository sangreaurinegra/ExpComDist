\subsection{Smart Judiciary System: A Smart Dust
            Based IoT Application
}

Este artículo "Smart Judiciary System: A Smart Dust
Based IoT Application"\cite[pág.128]{somaniEmerging2019}, fue elegido debido a que el concepto de tener sensores lo suficientemente pequeños como para llamarlos polvo, da para pensar en nuevas aplicaciones de tecnología IoT, y son una invitación a utilizar la imaginación.
El mismo plateó que la tecnología presentada mejoraría la precisión, transparencia e inteligencia de las posibles aplicaciones involucradas.
El artículo planteó como base la utilización de la tecnología de polvo inteligente para la detección de actividades criminales.
Buscando con su aplicación la reducción de crímenes. Lo que actualmente se puede apreciar en sistema de vigilancia mediante redes de cámaras dispuestas
para desalentar la actividad criminal, o sea con similares características pero utilizando estos sensores minúsculos.\par

Los autores mencionaron que los militares norteamericanos invierten en la investigación de tecnologías como las microelectromecanicas (MEMS) con aplicación en vigilancia. Plantean que en el futuro nubes de pequeños sensores podrán ser trasladadas en el aire y que otras aplicaciones de esta tecnología pueden ser la construcción de puentes, el seguimiento del clima, el monitoreo de tráfico, los estudios biológicos entre otras.\par

Se espera que estos sensores minúsculos puedan capturar imágenes de calidad, que puedan medir cualquier cosa permitiendo de esta manera tener un monitoreo de los ciudadanos de forma inalámbrica, aplicando tecnologías como la de reconocimiento facial, reconocimiento de gestos (presentado en otros artículos de los libros mencionados en esta sección), reconocimiento y grabación de emociones para la predicción del crimen (como planteó la película Minority Reports), detección de microbios coordinada por ejemplo con una central analizadora de datos perteneciente a la autoridad de salud, entre otras\par

Un modelo de una típica partícula inteligente fue descrito con sus diferentes componentes destacándose sensores MEMS, un diodo laser, espejo de haz, fuente de poder basada en baterías de film, celdas solares. Los datos son recolectados en  la partícula y trasmitidos a una base de control utilizando radiofrecuencia o trasmisión óptica.\par

En el artículo se platearon además diferentes estrategias de despliegue del polvo inteligente, la necesidad de un sistema que proteja la privacidad indicando la presencia de polvo inteligente.\par

