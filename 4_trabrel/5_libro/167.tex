
\paragraph{
    \textbf{\emph{Rating Prediction by Combining User Interest
    and Friendly Relationship}
    }
    \cite[pág. 167]{somani_emerging_2019}.
}

Dada la cantidad de información en  redes sociales y las interoperaciones personales.
Los autores sugirieron una aproximación para calificar predicciones,
combinando los intereses de los usuarios, la información de sus amistades 
utilizando un factor de reputación para mejorar la precisión.

Cuatro factores sociales fueron tomados en cuenta, como ser los intereses individuales,
las similitudes de los intereses entre usuarios, calificación del comportamiento entre usuarios, 
y la calificación de la difusión del comportamiento relacional. Con esos cuatro factores dentro
de un modelo de recomendaciones fue creada la matriz de factorización. 
 
Se realizaron una serie de experimentos considerando y no las relaciones sociales 
como así la influencia interpersonal y las decisiones personales. 
En otras palabras utilizando las decisiones de los contactos ponderando 
la influencia de los mismos como una forma de mejorar las predicciones.

Fue concluido que los resultados obtenidos validan la conveniencia del método propuesto. 