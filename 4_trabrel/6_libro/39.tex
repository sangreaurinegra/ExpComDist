\subsection{
    \textbf{\emph{An Efficient ATM Surveillance
            Framework Using Optical Flow
            with CNN}
    }
}

En el artículo "An Efficient ATM Surveillance
Framework Using Optical Flow
with CNN"\cite[pág. 39]{chaudharyMicroservices2020}, fue presentada una aproximación a un sistema que identifica movimientos sospechosos en zonas sensibles como ser cerca de cajeros ATM para alertar a la policía cercana para evitar tanto robos como perdidas humanas.
Utilizando cámaras desde diferentes ángulos lograron un flujo denso de imágenes a procesar, donde además de determinar si el comportamiento puede determinar una situación de peligro también realiza un reconocimiento facial de expresiones.
\par

Los autores utilizaron una red neuronal de convolución para machear sentencias del lenguaje natural con
características pre entrenadas sobre reconocimiento de objetos y estimación de posicionamiento.
La metodología para abordar el problema fue dividida en 5 partes, la detección del movimiento utilizando óptica fluida, el pasaje de esas capturas a la capa de convolución, la reducción de resolución de las capturas utilizando la capa de pooling, pasaje de las capturas a capas totalmente conectadas y la clasificación en clases.
\par

Los autores entrenaron el modelo utilizado Python y OpenCV y el método de Lucas-Kanade.
Resaltaron que la ventaja de dicho método es que en cuanto el objeto comienza
moverse en cualquier dirección con respecto al fondo estacionario, la dirección de
ese movimiento se agrega en marcos lo cual ayuda en la detección de la actividad que está ocurriendo.
\par

Como conclusiones se presentaron los porcentajes de la precisión promedio de los métodos siendo el mejor el de Lucas-Kanade (92,39\%) y el peor el método de utilización de óptica fluida (68,45\%).
\par
