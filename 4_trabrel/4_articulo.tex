\paragraph{
    \textbf{\emph{Automation of distributed data management in applied 
                    microservices package for scientific computations
            }
    }
    \cite{oparin_automation_2020}.
}

En el marco de la conferencia 
\textbf{The International Workshop on Information, Computation, and Control Systems for Distributed Environments}, 
fue presentado el artículo en cuestión. 

En el que se ofrece conjunto de herramientas especializadas para automatizar la gestión del conocimiento en
la creación de microservicios y la acumulación de datos, aplicado a 
cálculos científicos en un entorno informático híbrido.

Se enfocó en la utilización de microservicios para la automatización de la creación y adaptación de aplicaciones existentes
para proveer la habilidad de resolver problemas complejos dentro de algún subject domain (SD).

El concepto arquitectónico base de microservicios que se utilizó es Applied Microservice Package (AMP),
el cual según se indicó provee de un camino efectivo para la computación científica distribuida.

El modelo computacional AMP se encuentra representado por un conjunto de pequeñas piezas con bajo acoplamiento, 
implementado con microservicios.

La inteligencia de las AMP esta basada en el modelo de SD, 
que se comprende como una colección de información sobre los objetos SD y las relaciones entre ellos.

De esta manera el conjunto de herramientas desarrolladas automatiza la creación y actualización
de las bases locales de conocimiento del los agentes distribuidos. Las bases locales de conocimiento 
son creadas de mediante el uso de interfaces administradas por estos agentes. 
El sistema utiliza dicho conocimiento para testear las actualizaciones de los microservicios, 
además de proveer sincronización y persistencia de los datos calculados.
 
Por el hecho de plantear esta alternativa, que utiliza una arquitectura de microservicios automatizada y como parte de una 
lógica customizable es que se realiza la revisión de este artículo.


Se mencionó que el uso de arquitectura de agentes en conjunto con microservicios provee de una oportunidad práctica 
para implementar mecanismos de interacción semántica de los agentes en una red P2P. 

Las ventajas de las redes P2P mencionadas fueron escalabilidad, minimización de costos de comunicación, 
autoorganización adaptabilidad, tolerancia a fallos y autobalance de carga. 
De esta manera fue evitada la centralización del acceso a los de servicios.


Mediante el desarrollo de una solución que se basó en contratos Booleanos (BDS) para Sistemas Dinámicos, 
se validó la solución propuesta. 
BDS tiene aplicación en el campo de la bioinformatica.

Se diseñó un sistema que además de tener un administrador de autorización y manejo de archivos tiene:

\begin{itemize}
    \item Administración de sincronización, administra la sincronización de las bases de conocimientos locales.
    \item Wizard de Creación, automatiza la creación de nuevos servicios basados e en plantillas estándar,
     populando la base de conocimiento local.
    \item Wizard de Configuración, automatiza el alta de parámetros de los agentes, 
    para definir la relación de los agentes con las funcionalidades de los microservicios.
    \item Wizard de Test, automatiza el testeo de los paquetes desplegados, 
    y la interacción de los agentes con los microservicios.
    \item Wizard de Actualización, 
    automatiza la actualización de agentes y microservicios desde repositorios externos y locales.
\end{itemize}

Se accede a todo el paquete administrativo a través de una interfaz web.


Luego se explicitó el funcionamiento de la actualización y testeo de microservicios, 
en el que se aclaró que en caso de que el microservicio estuviera involucrado 
en un proceso en ejecución queda bloqueada la actualización hasta que esté se complete. 
De igual manera se bloquea el acceso a usuarios a procesos que involucren el microservicio a actualizar y testear. 

En cuanto al ejemplo práctico de la plataforma fue desarrollado AMP para el estudio cualitativo de BDS, 
el cual se dividió en los siguientes grupos de microservicios:

\begin{itemize}
    \item Creación de modelos Booleanos.
    \item Validación de los contratos Booleanos.
    \item Pre y postprocesador del procesamiento.
\end{itemize}

Se concluyó que el conjunto de herramientas provee varios modelos de datos,
sincronizando entre recursos locales y de la nube, microservicios actualizados y testeados. 
Las pruebas realizadas confirmaron la efectividad de la solución para resolver
problemas prácticos significativos de carácter científico. Como también que la automatización y administración 
mejora la productividad del desarrollador, creando y actualizando AMP, como también la del usuario final en la 
solución de problemas de investigación de BDS.
