\section{Objetivos}
El proyecto propone estudiar, diseñar y desarrollar implementaciones de software distribuido basadas en la arquitectura de microservicios para evaluar diferentes frameworks de desarrollo sobre dicha arquitectura.\par

Las principales metas del proyecto incluyen desarrollar un prototipo de procesamiento de grandes volúmenes de datos para evaluar los siguientes frameworks:
\begin{itemize}
    \item{Spring}
    \item{Quarkus}
    \item{Micronaut}
    \item{helidon.io}
\end{itemize}
y su integración con herramientas para mejorar su performance como GraalVM, herramienta de transformación a código nativo. Implementado en contenedores sobre Kubernetes.
Serán incluidos en dicha evaluación las ventajas que brinda cada herramienta para el desarrollo como así el desempeño de los mismos.
Se propone realizar un relevamiento del estado del arte de la temática y avanzar en una propuesta flexible y eficiente para evaluar las herramientas mencionadas.
Los productos de software se desarrollarán sobre la plataforma de computación científica de alto desempeño provista por el Centro Nacional de Supercomputación (Cluster-UY).\par

