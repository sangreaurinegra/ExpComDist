\chapter{Trabajo relacionado}

En este capitulo se incluye los trabajos de investigación relacionados, publicaciones, productos y libros al respecto de la utilización de Microservicios en arquitectura distribuidas para el procesamiento de grandes volúmenes de datos.


\section{Papers}

 An Experimental Study on Microservices based Edge Computing Platforms \cite{DBLP}.
 
https://arxiv.org/abs/2004.02372
Motivado por la arquitectura de microserviucios y su aplicaciona a  ciudades inteligentes
este paper investiga multiples politicas de deploy en plataformas de computacion de borde.
Para ello utiliza contenedores docker y a traves de pruebas experimentales e distintos escenarios, compara el desempenho de las mismas.
Considera que la capacidead de recolectar y procesar los datos de borde es la clave.
EL trabajp trata de respodner ;las preguntas 

Is it suitable to run multiple microservices inside one
container just considering the performance of the edge
device? Which type of microservices with different re-
source consuming inclination could be put together if the
answer is yes?
•Is the effect of interference that executes multiple mi-
croservices running on fog computing and edge comput-
ing scenarios different? and
•What is the trade-off between the “one process per
container” rule and the existing limitation of resource at
edge side, such as computation and memory? 

Argumenta porqeu la eleccion de contenedores y no VMs, haciendo referencia a estudios de IBM en cuanto aconsumo de recursos .
 The IBM
Research Division measured the performance of Docker in
terms of CPU, memory, disk I/O and compared the result
with KVM. ([7] W. Felter, A. Ferreira, R. Rajamony, and J. Rubio, “An updated perfor-
mance comparison of virtual machines and linux containers,” in 2015
IEEE international symposium on performance analysis of systems and
software (ISPASS). IEEE, 2015, pp. 171–172.)

Define como conveniete el utilizar un contenedor por microservicio, haciendo referencia a D. N. Jha, S. Garg, P. P. Jayaraman, R. Buyya, Z. Li, and R. Ranjan, “A
holistic evaluation of docker containers for interfering microservices,”
in 2018 IEEE International Conference on Services Computing (SCC).
IEEE, 2018, pp. 33–40 .



1) CPU performance: To evaluate the performance of the
CPU, we chose the Linpack benchmarks which measure
the system’s floating point computing power by solving
linear equations. And the capability of the CPU is
measured in terms of FLOPS (floating point operations
per second).
2) Memory performance:We chose the STREAM bench-
marks to measure the performance of the memory.
STREAM is a simple synthetic benchmark program that
measures sustainable memory bandwidth (in MB/s) and
the corresponding computation rate for simple vector
kernels.
3
3) Disk performance: To measure the disk I/O capability,
we chose the Bonnie++ benchmark which gives the
results in terms of input, output, rewrite (in Kb/s) and
seeks (in per second

 

\section{Libros}

Quarkus Cookbook Kubernetes-Optimized Java Solutions by Alex Soto, Jason Porter

Microservices in Big Data Analytics

https://www.springer.com/gp/book/9789811501272

Este libro recolecta papers de vanguardia que exploran principios, técnicas y aplicaciones de Microservicios para el análisis de grandes volúmenes de datos.


The ICETCE-2019 is the latest installment in a successful series of annual conferences that began in 2011. Every year since, it has significantly contributed to the research community in the form of numerous high-quality research papers. This year, the conference’s focus was on the highly relevant area of Microservices in Big Data Analytics.


Orquestación de servicios para el desarrollo de aplicaciones para big data
http://sedici.unlp.edu.ar/handle/10915/69999


Diseño de un prototipo de una arquitectura basada en microservicios para la integración de aplicaciones Web altamente transaccionales. Caso: Entidades financieras.
https://dspace.ups.edu.ec/handle/123456789/18532

https://searchdatamanagement.techtarget.com/feature/Microservices-and-big-data-start-to-get-closer

https://link.springer.com/article/10.1007/s11277-020-07822-0

https://blog.gramener.com/enterprise-microservices-big-data-applications/

\section{Productos Comerciales}