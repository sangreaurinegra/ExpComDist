\subsection{Patrones}

Existen una serie de patrones arquitectónicos en el ecosistema de microservicios.
Muchos de esos patrones son esenciales a la hora de diseñar una arquitectura de microservicios porque seran la herramientas
necesarias para desarrollar un sistema robusto, escalable, mantenible, monitoreable sin morir en el intento.

de
Microservices Patterns With examples in Java by Chris Richardson

Decomposition patterns
Decompose by business capability (51)
Decompose by subdomain (54)

Messaging style patterns
Messaging (85)
Remote procedure invocation (72)

Reliable communications patterns
Circuit breaker (78)
Service discovery patterns
3rd party registration (85)
Client-side discovery (83)
Self-registration (82)
Server-side discovery (85)

Transactional messaging patterns
Polling publisher (98)
Transaction log tailing (99)
Transactional outbox (98)

Data consistency patterns
Saga (114)

Business logic design patterns
Aggregate (150)
Domain event (160)
Domain model (150)
Event sourcing (184)
Transaction script (149)

Querying patterns
API composition (223)

Command query responsibility segregation
(228)

External API patterns
API gateway (259)
Backends for frontends (265)

Testing patterns
Consumer-driven contract test (302)
Consumer-side contract test (303)
Service component test (335)

Security patterns
Access token (354)

Cross-cutting concerns patterns
Externalized configuration (361)
Microservice chassis (379)

Observability patterns
Application metrics (373)
Audit logging (377)
Distributed tracing (370)
Exception tracking (376)
Health check API (366)
Log aggregation (368)

Deployment patterns
Deploy a service as a container (393)
Deploy a service as a VM (390)
Language-specific packaging format (387)
Service mesh (380)
Serverless deployment (416)
Sidecar (410)

Refactoring to microservices patterns
Anti-corruption layer (447)
Strangler application (432)
