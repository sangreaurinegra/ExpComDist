\section{Microservicios}

Here are just a few of the enterprise benefits of microservices.

    Independently deployable. ...
    Right tool for the job. ...
    Precise scaling. ...
    There are challenges, too. ...
    Containers, Docker, and Kubernetes. ...
    API gateways. ...
    Messaging and event streaming. ...
    Serverless.
    
    https://www.ibm.com/cloud/learn/microservices
    
    
    
Because of many attractive features, such as good scalabil-
ity, fine granularity, loose coupling, continuous development,
and low maintenance cost, the microservices architecture has
emerged and gained a lot of interests both in industry and
academic community
Compared to traditional SOAs
in which the system is a monolithic unit, the microservices
architecture divides an monolithic application into multiple
atomic microservices that run independently on distributed
computing platforms

ach microservice performs one specific
sub-task or service, which requires less computation resource
and reduces the communication overhead. Such characteristics
make the microservices architecture an ideal candidate to
build a flexible platform, which is easy to be developed and
maintained for cross-domain applications


Container technology offers a more lightweight method to
abstract the applications from the system environment which
allows the microservices to be deployed quickly but also
consistently. Compared to VMs, containers not only provide
all the libraries and other dependencies but also consume less
resource and produce lower overhead
    