\section{Microservicios}

\paragraph{
    Los microservicios son componentes de software atómicos funcionalmente e independientes de los demás servicios del sistema. La independencia es vertical y aunque puede consumir y ser consumido por otros servicios (micro o no), solo se debe cumplir con los contratos de interfaz para sustituirlo por otro. De un microservicio solo es necesario conocer que parámetros que toma, que resultado y el formato devuelve, y la ubicación para consumirlo.
    Debido a su independencia técnica cada microservicio puede ser implementado por diferentes tecnologías como, lenguajes de programación, bases de datos, servidores de aplicaciones, servidores web, entre otros. Los microservicios pueden ser ubicados en distintos lugares ya sea por motivos físicos o técnicos, de acuerdo a las necesidades planteadas. Un ejemplo de motivos físicos sería que la fuente de datos que debe consumir el microservicio se encuentre en un lugar lejano, en este caso es preferible que el microservicio que la utilice este cerca de la fuente. Estando cerca de la fuente el microservicio no solo puede acceder a los datos de manera mas rápida sino que al retornar datos resumidos podría acelerar el procesamiento de un sistema con un componente central que utilice los datos retornados (ejemplos IoT, edge). En cuanto a los motivos técnicos para la ubicación de los microservicios se puede indicar que el hecho de que puedan ser implementados por distintas tecnologías, permite que se puedan implementar para dispositivos de bajos recursos como pueden ser los dispositivos edge.
}

\paragraph{
    Las arquitecturas basadas en microservicios, son arquitecturas cuyos componentes son microservicios levemente acoplados, independientemente distribuibles y con un cometido funcional muy especifico. 
    Las mismas se aprovechan de las ventajas de los microservicios para destacarse como arquitecturas más dinámicas sobre otras monolíticas.
    Algunas de las ventajas son la independencia a la hora de distribuir que permite actualizarlos sin actualizar toda la aplicación, poder utilizar la herramienta correcta para el problema especifico, escalamiento preciso al poder escalar los microservicios que son necesarios y no toda app.
    También se dice que las arquitecturas de microservicios son una aproximación a las arquitecturas nativas de la nubes de TODO
}

-- formato de comunicacion communicate with one another over a combination of REST APIs, event streaming, and message brokers; and

-- capas are organized by business capability, with the line separating services often referred to as a bounded context.

Here are just a few of the enterprise benefits of microservices.

Independently deployable. ...
Right tool for the job. ...
Precise scaling. ...
There are challenges, too. ...
Containers, Docker, and Kubernetes. ...
API gateways. ...
Messaging and event streaming. ...
Serverless.

https://www.ibm.com/cloud/learn/microservices



Because of many attractive features, such as good scalabil-
ity, fine granularity, loose coupling, continuous development,
and low maintenance cost, the microservices architecture has
emerged and gained a lot of interests both in industry and
academic community
Compared to traditional SOAs
in which the system is a monolithic unit, the microservices
architecture divides an monolithic application into multiple
atomic microservices that run independently on distributed
computing platforms

ach microservice performs one specific
sub-task or service, which requires less computation resource
and reduces the communication overhead. Such characteristics
make the microservices architecture an ideal candidate to
build a flexible platform, which is easy to be developed and
maintained for cross-domain applications


Container technology offers a more lightweight method to
abstract the applications from the system environment which
allows the microservices to be deployed quickly but also
consistently. Compared to VMs, containers not only provide
all the libraries and other dependencies but also consume less
resource and produce lower overhead







Microservicios vs SOA (serviceoriented architecture)
The differences between microservices and SOA can be a bit less clear. While technical contrasts can be drawn between microservices and SOA, especially around the role of the enterprise service bus (ESB), it’s easier to consider the difference as one of scope. SOA was an enterprise-wide effort to standardize the way all web services in an organization talk to and integrate with each other, whereas microservices architecture is application-specific.

https://www.ibm.com/cloud/blog/soa-vs-microservices


****

de https://www.leanix.net/en/blog/benefits-of-using-microservices-with-big-data-applications

Application Scalability

Traditional monolithic applications simply do not offer the flexibility that applications built on microservices do. An application running on microservices have each supporting component running on different servers, which can be scaled up or down whenever needed. Big Data systems tend to process a lot of data at a higher speed than applications ran on monolithic frameworks can handle.
Data Consistency I\& Quality

Big Data increases the velocity and volume of data being processed at one time on a server. It also increases the variety and veracity of uncertain data. As data volume increases, it is important to keep an eye on the data quality. For example, A data error on the Nasdaq exchange caused chaos recently, as the introduction of test data into the live systems vastly impacted the shares prices of a number of tech companies. Two notable examples: Amazon prices crashed by 87%, while Zynga bounced up by an amazing 3,292%. In this case, the error can be linked directly to data quality.

Applications built on microservices are easier to maintain, test, and scale than monolithic applications.

Ease of Code Modification

Microservice frameworks allow for different employees specializing in various coding languages to modify code. This will add direct benefits to your organization, mainly including diversifying and strengthening your talent pool.

